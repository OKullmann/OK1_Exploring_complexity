\documentclass{report}

% \usepackage[active]{srcltx}

\usepackage{a4}
\usepackage[all]{xy}
\usepackage{enumerate}
\usepackage{xr}
\usepackage{theorem}
\usepackage[driverfallback=hypertex]{hyperref}

\usepackage{float}
\usepackage{bussproofs}

%
\scrollmode
%
\usepackage{amsmath}
\usepackage{amsfonts}
\usepackage{amssymb}
\usepackage{latexsym}
\usepackage{stmaryrd}
\usepackage{array}
\usepackage{exscale}

\renewcommand{\thefootnote}{\arabic{footnote})}%
%
\newtheorem{defi}{Definition}[section]
\newtheorem{lem}[defi]{Lemma}
\newtheorem{thm}[defi]{Theorem}
\newtheorem{corol}[defi]{Corollary}
\newtheorem{propo}[defi]{Proposition}
\newtheorem{exerc}[defi]{Exercise}
\newtheorem{conj}[defi]{Conjecture}
\newtheorem{examp}[defi]{Example}
\newtheorem{quest}[defi]{Question}
\newtheorem{spec}[defi]{Speculation}
%\newtheorem{claim}[defi]{Claim}

\newcommand{\nc}{\newcommand}

\newcommand{\DST}{\displaystyle}
\newcommand{\TST}{\textstyle}
\newcommand{\SST}{\scriptstyle}
\newcommand{\SSST}{\scriptscriptstyle}

\nc{\bm}{\boldmath}
\nc{\bmm}[1]{\mbox{\bm$\DST #1$}}% Math. Fettdruck im Text- wie im math. Modus

\newcommand{\mb}{{\:|\:}} % Mengenbildner; set creation
\newcommand{\set}[1]{\{ #1 \}}
\newcommand{\setb}[1]{\big \{ \, #1 \, \big \}}
\newcommand{\tb}[2]{\set{#1, \dots, #2}} % Teilbereich

\newcommand{\ol}{\overline}
\newcommand{\ul}{\underline}
\newcommand{\es}{\emptyset}
\newcommand{\sm}{\setminus}
\newcommand{\ve}{\varepsilon}
\newcommand{\vp}{\varphi}
\newcommand{\bw}{\bigwedge}
\newcommand{\bv}{\bigvee}
\newcommand{\bc}{\bigcup}
\newcommand{\bca}{\bigcap}
\newcommand{\Lra}{\Leftrightarrow}
\newcommand{\Lora}{\Longrightarrow}
\newcommand{\Lla}{\Longleftarrow}
\newcommand{\Llra}{\Longleftrightarrow}
\newcommand{\Ra}{\Rightarrow}
\newcommand{\La}{\Leftarrow}
\newcommand{\ra}{\rightarrow}
\newcommand{\lora}{\longrightarrow}
\newcommand{\la}{\leftarrow}
\newcommand{\lra}{\leftrightarrow}
\newcommand{\da}{\downarrow}
\newcommand{\ub}{\underbrace}
\newcommand{\ob}{\overbrace}
\newcommand{\sst}{\subset}
\newcommand{\sse}{\subseteq}
\newcommand{\spt}{\supset}
\newcommand{\spe}{\supseteq}
\newcommand{\fa}{\forall}
\newcommand{\ex}{\exists}
\newcommand{\mr}{\mathrm}
\newcommand{\mc}{\mathcal}
\newcommand{\mf}{\mathfrak}
\newcommand{\vtr}{\vartriangleright}
\newcommand{\trd}{\triangledown}
\newcommand{\DMO}{\DeclareMathOperator}
%
\newcommand{\ZZ}{\mathbb{Z}}
\newcommand{\NN}{\mathbb{N}}
\newcommand{\NNZ}{\NN_0}
\newcommand{\QQ}{\mathbb{Q}}
\newcommand{\RR}{\mathbb{R}}
\newcommand{\CC}{\mathbb{C}}
\newcommand{\PP}{\mathbb{P}}
\newcommand{\AAM}{\mathbb{A}}
\newcommand{\MM}{\mathbb{M}}
\newcommand{\GG}{\mathbb{G}}
\newcommand{\BB}{\mathbb{B}}
\newcommand{\KK}{\mathbb{K}}
\newcommand{\EE}{\mathbb{E}}
\newcommand{\FF}{\mathbb{F}}
\newcommand{\WW}{\mathbb{W}}
\newcommand{\LL}{\mathbb{L}}
\newcommand{\HH}{\mathbb{H}}
\newcommand{\UU}{\mathbb{U}}
\newcommand{\OO}{\mathbb{O}}
\newcommand{\SSM}{\mathbb{S}}
\newcommand{\DD}{\mathbb{D}}
\newcommand{\TT}{\mathbb{T}}

\newcommand{\Va}{\mc{V\hspace{-0.1em}A}}
\newcommand{\Lit}{\mc{LIT}}
\newcommand{\Cl}{\mc{CL}}
\newcommand{\Cls}{\mc{CLS}}

\DeclareMathOperator{\lit}{lit}
\DeclareMathOperator{\var}{var}

\DeclareMathOperator{\res}{\diamond} % die (partielle) Resolutionsoperation

% Schubfachformeln (pigeonhole formulas):
\newcommand{\php}{\mathrm{PHP}}
\newcommand{\fphp}{\mathrm{FPHP}} % mit AMO-Klauseln
\newcommand{\ophp}{\mathrm{OPHP}} % jedes Fach verwendet; every hole is used
\newcommand{\ofphp}{\mathrm{BPHP}} % bijektive Form
\newcommand{\ephp}{\mathrm{EPHP}} % PHP with extended-resolution clauses from Cook1976short



\begin{document}

\title{The resolution complexity of Pigeonhole formulas}

\author{
  Jonathan Richards\\
  \href{http://www.swan.ac.uk/compsci/}{Computer Science Department}, \href{http://www.swan.ac.uk/science/}{College of Science}\\
  \href{http://www.swan.ac.uk/}{Swansea University, UK}\\
  {- url}
}

\maketitle

\begin{abstract}
  We study the \emph{precise} complexity of Pigeonhole formulas $\php^{n+1}_n$ for tree- as well as dag-resolution.
\end{abstract}

\tableofcontents

\chapter{Introduction}
\label{cha:Introduction}

XXX

\section{History}
\label{sec:History}

XXX Haken in his seminal paper \cite{Haken1985Intractability} for the first time showed an exponential lower bound for \emph{full} resolution.



\chapter{Preliminaries}
\label{cha:Preliminaries}

\section{Clause-sets}
\label{clause-sets}

We follow the general notations and definitions as outlined in \cite{Kullmann2007HandbuchMU}. We use $\NN = \set{1,2,\dots}$ and $\NNZ = \NN \cup \set{0}$.

Let $\Va$ be the set of variables, and let $\Lit = \Va \cup \set{\ol{v} : v \in \Va}$ be the set of literals, the disjoint union of variables as positive literals and complemented variables as negative literals. The complementation operation is extended to a (fixed point) free involution on $\Lit$, that is, for all $x \in \Lit$ we have $\ol{\ol{x}} = x$. We assume $\NN \sse \Va$, with $\ol{n} = -n$ for $n \in \NN$ (whence $\ZZ \sm \set{0} \sse \Lit$).\footnote{This yields a convenient way of writing down examples, where we do not have to distinguish between different types of variables, and thus can just use natural numbers as variables. Furthermore the set of variables is infinite, and thus is never exhausted by a clause-set.}

We use $\ol{L} := \set{\ol{x} : x \in L}$ to complement a set $L$ of literals. A clause is a finite subset $C \subset \Lit$ which is complement-free, i.e., $C \cap \ol{C} = \es$; the set of all clauses is denoted by $\Cl$. A clause-set is a finite set of clauses, the set of all clause-sets is $\Cls$. By $\var(x) \in \Va$ we denote the underlying variable of a literal $x \in \Lit$, and we extend this via $\var(C) := \set{\var(x) : x \in C} \subset \Va$ for clauses $C$, and via $\var(F) := \bc_{C \in F} \var(C)$ for clause-sets $F$. The ``possible'' literals for a clause-set $F$ are denoted by $\lit(F) := \var(F) \cup \ol{\var(F)}$, while for the actually occurring literals we just use ordinary union $\bc F \subset \Lit$.


\section{Resolution}
\label{sec:Resolution}

Two clauses $C, D \in \Cl$ are resolvable iff they clash in exactly one literal $x$, that is, $C \cap \ol{D} = \set{x}$, in which case their resolvent is $\bmm{C \res D} := (C \cup D) \sm \set{x,\ol{x}}$ (with resolution literal $x$). A resolution tree is a full binary tree formed by the resolution operation. We write \bmm{T : F \vdash C} if $T$ is a resolution tree with axioms (the clauses at the leaves) all in $F$ and with derived clause (at the root) $C$.


\section{Pigeonhole principles}
\label{sec:php}

The \textbf{pigeon-hole principle} states that there is an injective map from $\tb1m$ to $\tb1k$ for $m,k \in \NNZ$ iff $m \le k$. So when putting $m$ pigeons into $k$ holes, if $m > k$ then at least one hole must contain two or more pigeons. We formalise the pigeon-hole principle as a clause-set \bmm{\php^m_k}, which is unsatisfiable iff $m > k$.
\begin{defi}\label{def:php}
  We use variables $p_{i,j} \in \Va$ for $i, j \in \NN$ such that $(i,j) \not= (i',j') \Ra p_{i,j} \not= p_{i',j'}$. For $m, k \in \NNZ$ let
  \begin{eqnarray*}
    F^{\ge 1} & := & \setb{\set{p_{i,j} \mb j \in \tb1k}}_{i \in \tb1m }\\
     F_{\ge 1} & := & \setb{\set{p_{i,j} \mb i \in \tb 1m}}_{j \in \tb 1k }\\
    F^{\le 1} & := & \setb{\set{\ol{p_{i,j_1}},\ol{p_{i,j_2}}} \mb i \in \tb 1m, j_1, j_2 \in \tb 1k, j_1 \not= j_2}\\
    F_{\le 1} & := & \setb{ \set{\ol{p_{i_1,j}},\ol{p_{i_2,j}}} \mb i_1,i_2 \in \tb1m, i_1 \not= i_2, j \in \tb1k }.
  \end{eqnarray*}
  The \textbf{pigeon-hole clause-set} $\bmm{\php^m_k} \in \Cls$ for $m,k \in \NNZ$ uses variables $p_{i,j}$ for $i \in \tb 1m$, $j \in \tb 1k$, and is defined, together with the \textbf{functional}, \textbf{onto}, and \textbf{bijective} form, as
  \begin{eqnarray*}
    \bmm{\php^m_k} & := & F^{\ge 1} \cup F_{\le 1} \in \Cls\\
    \bmm{\fphp^m_k} & := & \php^m_k \cup F^{\le 1} \in \Cls\\
    \bmm{\ophp^m_k} & := & \php^m_k \cup F_{\ge 1} \in \Cls\\
    \bmm{\ofphp^m_k} & := & \php^m_k \cup F^{\le 1} \cup F_{\ge 1} \in \Cls.
  \end{eqnarray*}
\end{defi}

\begin{examp}\label{exp:PHPdef}
  PHP-clause-sets for small parameters:
  \begin{enumerate}
  \item $\php^0_0 = \top$, $\fphp^0_0 = \top$, $\ophp^0_0 = \top$, $\ofphp^0_0 = \top$.
  \item $\php^1_0 = \set{\bot}$, $\fphp^1_0 = \set{\bot}$, $\ophp^1_0 = \set{\bot}$, $\ofphp^1_0 = \set{\bot}$.
  \item $\var(\php^2_1) = \set{p_{1,1},p_{2,1}}$, $\php^2_1 = \set{\set{p_{1,1}}, \set{p_{2,1}}, \set{\ol{p_{1,1}},\ol{p_{2,1}}}}$, $\fphp^2_1 = \php^2_1$, $\ophp^2_1 = \php^2_1 \cup \set{\set{p_{1,1},p_{2,1}}}$, $\ofphp^2_1 = \ophp^2_1$.
  \item $\var(\php^3_2) = \set{p_{1,1},p_{2,1}, p_{3,1},p_{1,2}.p_{2,2}.p_{3,2}}$:
    \begin{eqnarray*}
      \php^3_2 & = & \set{\set{p_{1,1},p_{1,2}},\set{p_{2,1},p_{2,2}},\set{p_{3,1},p_{3,2}},
      \\
      && \set{\ol{p_{1,1}},\ol{p_{2,1}}},\set{\ol{p_{1,1}},\ol{p_{3,1}}},\set{\ol{p_{2,1}},\ol{p_{3,1}}},
      \\
      && \set{\ol{p_{1,2}},\ol{p_{2,2}}},\set{\ol{p_{1,2}},\ol{p_{3,2}}},\set{\ol{p_{2,2}},\ol{p_{3,2}}} }\\
      \fphp^3_2 & = &\\
      \ophp^3_2 & = &\\
      \ofphp^3_2 & = &.
    \end{eqnarray*}
  \end{enumerate}
\end{examp}

\begin{figure}[h]
  \centering
  \begin{tabular}{c|c|c|c|c}
    \hline
    n & $n$ &  $c$ & min \# nodes & min \# leaves \\ \hline
    0 & $0$ & $1$ & $1$ & $1$\\ \hline
    1 & $?$ & $?$ & $5$ & $3$    \\ \hline
    2 &                 &        \\ \hline
    3 &                 &        \\ \hline
    4 &                 &        \\ \hline
    \end{tabular}
  \caption{Basic meaurements for $\php^{n+1}_n$, $n \in \NNZ$}
  \label{fig:basicmeasurement}
\end{figure}

For \bmm{\php^m_k}, $m, k \in \NNZ$, the number of long clauses is $c(F^{\ge 1}) = m$, the number of short clauses is $c(F_{\le 1}) = \frac {n(n+1)n}{2}$, since we have a double-nested loop selecting two different cells in the same column with selection order being irreverent. 


\section{Matrix representation of the Pigeonhole principles}
\label{sec:phpMatrix}

Some further notation used in this paper include clause sets represented as a Matrix,
For \bmm{\php^m_k} the number of rows would be m, the number of pigeons.
Columns would be n, the number of holes, such that: $\set{p_{1,1},p_{1,2}}.$ from $\php^3_2$,  pigeon is in hole 1 or pigeon is in hole 2, from $\php^3_2$ would be:

\begin{displaymath}
  \set{p_{1,1},p_{1,2}} \leadsto
  \begin{pmatrix}
    +1 & +1\\
    0 & 0 &\\
    0 & 0
  \end{pmatrix} \leadsto
  \begin{array}{cc}
    + & +\\
    \\
    \\
  \end{array}
  \begin{pmatrix}
    + & +&\\
     &  &\\
     &  &
  \end{pmatrix} ???
\end{displaymath}
\[ \set{p_{1,1},p_{1,2}} = \left| \begin{array}{ccc}
+1 & +1 \\
 0 & 0 \\
 0 & 0 \end{array} \right| \] 
 
 A +1 value at the position 1,1 would represent the clause $\set{p_{1,1}}$ a 0 would represent no clause at that position, and a -1 would be a negated clause such as $\set{\ol{p_{1,1}}}$
 \
 To simplify these a +1 can be shown as a +, a 0 as a blank and -1 as a -.
 \[ \php^3_2 =
 \left| \begin{array}{ccc}
 + & + \\
   &   \\
   &   \end{array} \right|
 \left| \begin{array}{ccc}
   &   \\
 + & + \\
   &   \end{array} \right|
 \left| \begin{array}{ccc}
   &   \\
   &   \\
 + & + \end{array} \right|
 \left| \begin{array}{ccc}
 - &   \\
 - &   \\
   &   \end{array} \right|
 \left| \begin{array}{ccc}
 - &   \\
   &   \\
 - &   \end{array} \right|
 \left| \begin{array}{ccc}
   &   \\
 - &   \\
 - &   \end{array} \right|
 \left| \begin{array}{ccc}
   & - \\
   & - \\
   &   \end{array} \right|
 \left| \begin{array}{ccc}
   & - \\
   &   \\
   & - \end{array} \right|
 \left| \begin{array}{ccc}
   &   \\
   & - \\
   & - \end{array} \right|
\] 
 
Example of resolution, on a matrix
\[ \left| \begin{array}{ccc}
   &   \\
   &   \\
 + & + \end{array} \right| \res
 \left| \begin{array}{ccc}
 - &   \\
   &   \\
 - &   \end{array} \right|=
  \left| \begin{array}{ccc}
  - &   \\
    &   \\
    & + \end{array} \right|
  \]



\chapter{Literature review}
\label{cha:Literature}

\section{Haken: The Intractability of Resolution}
\label{sec:Haken1985Intractability}
\begin{itemize}
  \item 1985
  \item Exponential lower bound for full resolution.
  \item Using DNF rather than CNF, tautology.
  \item Used x pigeons to x+1 holes
  \item Example in matrix/arrays.
\end{itemize}
  \cite{Haken1985Intractability}

\section{Dantchev: Resolution and the Weak Pigeonhole Principle}
\label{sec:DantchevTreePHP}
\begin{itemize}
  \item 2001
  \item Optimal decision tree for PHP
  \item Lower and upper bound proofs
  \item Lower bound tree
\end{itemize} 
\cite{DantchevTreePHP}

\section{Buss: Resolution Proofs of Generalized Pigeonhole Principles}
\label{sec:Buss88GeneralProof}
\begin{itemize}
  \item 1988
  \item Focus on lower bound
  \item DNF form
\end{itemize}
\cite{Buss88GeneralProof}

\section{Buss: Tree Resolution Proofs of the Weak Pigeon-Hole Principle}
\label{sec:Buss96resolution}
\begin{itemize}
  \item 1996
  \item Upper and lower bound proof
  \item CNF form
\end{itemize}
\cite{Buss96resolution}



\section{Atserias: Improved Bounds on the Weak Pigeonhole Principle and Infinitely Many Primes from Weaker Axioms}
\label{sec:AtseriasImprovedBounds}
\begin{itemize}
  \item 2001
  \item From a far more mathematical paper
\end{itemize}
\cite{AtseriasImprovedBounds}


\section{Ojakian: Upper and lower Ramsey bounds in bounded arithmetic.}
\label{sec:Ojakian05UpperLower}
\begin{itemize}
  \item 2005
  \item From a far more mathematical paper
\end{itemize}
\cite{Ojakian05UpperLower}

\bibliographystyle{plain}

\bibliography{Literature}

\end{document}


%%% Local Variables: 
%%% mode: latex
%%% TeX-master: t
%%% End: 
